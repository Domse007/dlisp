% Created 2022-05-03 Tue 21:02
% Intended LaTeX compiler: pdflatex
\documentclass[11pt]{article}
\usepackage[utf8]{inputenc}
\usepackage[T1]{fontenc}
\usepackage{graphicx}
\usepackage{longtable}
\usepackage{wrapfig}
\usepackage{rotating}
\usepackage[normalem]{ulem}
\usepackage{amsmath}
\usepackage{amssymb}
\usepackage{capt-of}
\usepackage{hyperref}
\usepackage[germanb]{babel}
\usepackage{mhchem}
\author{Dominik Keller, G4a}
\date{}
\title{Rückblick}
\hypersetup{
 pdfauthor={Dominik Keller, G4a},
 pdftitle={Rückblick},
 pdfkeywords={},
 pdfsubject={},
 pdfcreator={Emacs 29.0.50 (Org mode 9.5.2)}, 
 pdflang={Germanb}}
\begin{document}

\maketitle

\section{Produkt}
\label{sec:orgfc212af}
Das Produkt ist auf Github gehostet und unter dieser URL erreichbar:
\href{https://www.github.com/domse007/dlisp}{https://www.github.com/domse007/dlisp}. Das finale Produkt befindet
sich auf dem \texttt{master} Branch.

\section{Anforderungen}
\label{sec:org97351aa}
\noindent
Das Ziel ist es eine möglichst vollständigen Lisp Core implementierung zu schreiben. Dieser Umfasst das Generieren eines AST und die anschliessende Ausführung des Trees. Core soll bedeuten, dass die wichtigsten Lisp Funktionen verfügbar sind, welche alle in Rust geschrieben sind. Das bedeutet, es soll keine Standard Library entstehen.\\

\noindent
Der Evaluator soll Variablen unterstützen, weshalb dieser auch Special Forms unterstützen musss. Special Forms ist ein Sammelbegriff für interne Funktionen, die aussehen wie Lisp Funktionen, aber keine sind.\\

\noindent
Die Implementierung soll ohne externe Crates kompilierbar sein. Zudem soll das ganze Crate ohne unsicherer (Note: \texttt{unsafe} Keyword) Code geschrieben werden.

\section{Implementierung}
\label{sec:orgbc15bbc}
\noindent
Die Lisp Implementierung funktioniert in den meisten Fällen. Es gibt
allerdings noch einige Probleme. Das grösste ist sicherlich, dass es
keinen Macro Support gibt.

\subsection{Eval}
\label{sec:orge6a9af2}
\noindent
Eval ist eine Rust Funktion, welche sich rekursiv aufruft und nach und
nach alles auflöst, bis sie \textbf{einen} Rückgabewert produziert hat. Das
erste, was eine Funktion überprüft ist, ob die Eingabe gequoted ist.
Gequoted heisst, dass der Rückgabewert dem Eingabewert ist und somit
die Daten nicht verändert werden. Danach wird der Typ des \texttt{LispObjects}
überprüft. Wenn es ein Symbol ist, wird überprüft ob der Manager eine
Variable besitzt. Falls ja, wird das \texttt{Symbol} mit dem Wert ersetzt.
Interessant wird es bei einer liste. Listen sind, wenn sie nicht
gequoted sind, Funktionsaufrufe, wobei das erste Element der
Funktionsname und der Rest die Argumente für die Funktion sind. Für
die Ausführung muss allerdings überprüft werden, auf welchem Level die
Funktion defininiert wurde:
\begin{enumerate}
\item \texttt{Rust Core}: Funktionen wie Additionen und Subtraktionen sind in Rust
implementiert und verhalten sich wie Funktionen die in Lisp
geschrieben sind.
\item \texttt{Special Forms}: Gewisse Funktionen können nicht in der standard
Lisp Funktionsweise definiert werden. Diese fallen unter die
Special Forms. Auch diese müssen in Rust geschrieben werden. Die
Folgenden Special Forms wurden implementiert:
\begin{enumerate}
\item \texttt{defun}: Defun definiert eine Funktion. Diese wird im Manager als
Key-Value Pair gespeichert, wobei der Key der Funktionsname ist
und der Body die Value.
\item \texttt{set}: Set nutzt den manager und setzt ein Key-Value Pair. Für
diese Operation wird der Manager genutzt, welcher für normale
Funktionen nicht zur Verfügung steht.
\item \texttt{quote}: Diese Special Form setzt das \texttt{Quoteflag}.
\end{enumerate}
\item \texttt{Lisp Funktionen}: Diese Funktionen werden durch \texttt{defun} gesetzt. Und
genau da ist der Fehler, denn defun müsste keine special Forms
sein, sondern im besten Fall ein Makro. In der aktuellen
Implementierung funktioniert es mehrheitlich. Der Rückgabewert
einer Funktion ist dabei das Resultat des letzten \texttt{eval} Aufrufs.
Probleme gibt es zur Zeit noch bei den Parmatern. Diese sind aus
zeitlichen Gründen noch nicht implementiert.
\end{enumerate}

\section{Rückblick}
\label{sec:org5853dc5}
\noindent
Es war vom Zeitaufwand, vor allem in dieser Zeit, ein doch viel
stressigeres Projekt, als ich anfangs angenommen habe. Schlussendlich
habe ich die letzten Bugs in den letzten Stunden vor der Abgabe
beheben können, weshalb ich nicht ausführöich testen konnte, ob es
Bedingungen gibt (und wenn ja wieviele?), auf die der Interpreter
aufgibt.\\

\noindent
Zudem hatte ich nicht die Zeit um die Errors umzustellen. Begonnen
habe ich mit \texttt{\&'static str} als Error Typ, denn diese sind schnell
geschrieben und trotzdem hilfreich beim Debuggen. In einem weiteren
Schritt würde es darum gehen auf \texttt{LispError} umzusteigen, denn der
sollte für den Nutzer deutlich aufschlussreicher sein, denn er erlaubt
es mehr Informationen zu speichern und ausgeben.\\

\noindent
Das Projekt nutzt zudem keine \texttt{unsafe} Codeblöcke. Das wenn im nächsten
Schritt das Error Handling finalisiert wird, kann das Programm nicht
mehr abstürzen.\\

\noindent
Schlussendlich bin ich aber ziemlich zufrieden mit der
Implementierung, denn ich habe alles ohne externe Tutorials oder
anderen Resourcen entwickelt. Alles was ich brauchte war \texttt{Emacs} mit
\texttt{lsp-mode} als Interface für \texttt{rust-analyzer} und den Compiler für Tipps
und Fehler.\\

\noindent
Abschliessend heisst das, dass alle am Anfang definierten Punkte
erfüllt wurden.\\
\end{document}
